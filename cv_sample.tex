%%
%% this resume has to be built using XeLaTex
%%

\documentclass[letterpaper]{soragna-onepage-twocols} % Use US Letter paper, change to a4paper for A4 

%----------------------------------------------------------------------------------------
%	SETTINGS
%----------------------------------------------------------------------------------------

% color used for all standard text
\colorlet{PrimaryColor}{SlateGray}

% color used for all section headers
% options: Orange, Sepia, LightGray
\colorlet{HeadingsColor}{Orange}

% font used for main text
% options: Lato, Roboto, Roboto_mono
\choosefont{Roboto}

% icons style for title and sections
% options: fontawesome, freepik, or leave it empty
\chooseiconset{freepik}

% use or not logo for positions (schools and companies)
% options: true or leave it empty
\usepositionlogos{true}

%----------------------------------------------------------------------------------------
%	DOCUMENT
%----------------------------------------------------------------------------------------

\begin{document}

\globalcolor{PrimaryColor}

\nocite{*}

%----------------------------------------------------------------------------------------
%	TITLE SECTION
%----------------------------------------------------------------------------------------

\lastupdated % Print the Last Updated text at the top right

\headersection{
    name={John},
    surname={Doe},
    mail={john.doe@gmail.com},
    website={https://www.whitehouse.gov/},
    address={Address}
}

%----------------------------------------------------------------------------------------
%	LEFT COLUMN
%----------------------------------------------------------------------------------------

\begin{minipage}[t]{0.3\textwidth} % The left column takes up 33% of the text width of the page

%------------------------------------------------
% About Me
%------------------------------------------------

\leftsection{About Me}{\aboutmeicon}

Current Co-founder \& Software Engineer in start-up company PLAT Corp. 
5+ years experience specializing in backend/infrastructure, web development and computer security. 
Super nerd who loves Vim, Linux and OS X and enjoys to customize all of the development environment. 
Interested in devising a better problem-solving method for challenging tasks, and learning new technologies and tools if the need arises.

\sectionspace % Some whitespace after the section

%------------------------------------------------
% Languages
%------------------------------------------------

\leftsection{Languages}{\languagesicon}

\languageskill{Italian}{Mothertongue}{images/flag/italy.png}
\languageskill{English}{Professional proficiency}{images/flag/united-states.png}
\languageskill{Chinese}{Basic spoken}{images/flag/china.png}

\vspace{-12pt} % For some reason there is a big space between this section and the following, so reduce it
\sectionspace % Some whitespace after the section

%------------------------------------------------
% Skills
%------------------------------------------------

\leftsection{Skills}{\technicalskillsicon}

\subsection{Tech skills}

3+ years experience with \textbf{C++}\\
{\emph{Libraries:} \small\emph{Eigen, OpenCV, ROS}}\\
3+ years experience with \textbf{Python}\\
{\emph{Libraries:} \small\emph{Numpy, Scikit-learn, Tensorflow}}\\
Worked on several projects using \textbf{Bash}, \textbf{C}, \textbf{Java}, \textbf{JavaScript},  \textbf{Matlab}\\
Daily user of \textbf{Docker}, \textbf{Git}, \textbf{LaTeX}\\

\subsection{Soft skills}

Fast Learner\\
\emph{I'm always curious and eager to learn new concepts in any subject I encounter.}\\
Problem Solver\\
\emph{My objective-driven mindset allows me to quickly find scalable solutions to everyday issues.}\\
Independent\\
\emph{I'm capable of working and of organizing duties with small or no supervision.}\\
Communicator\\
\emph{I have done several publich speeches and created effective presentation slides.}\\

%----------------------------------------------------------------------------------------

\end{minipage} % The end of the left column
\hfill
%
%
%----------------------------------------------------------------------------------------
%	RIGHT COLUMN
%----------------------------------------------------------------------------------------
%
%
\begin{minipage}[t]{0.66\textwidth} % The right column takes up 66% of the text width of the page

%------------------------------------------------
% Experience
%------------------------------------------------

\rightsection{Experience}{\experienceicon}

\createposition{
    company={COURSERA},
    title={KPCB Fellow + Software Engineering Intern},
    date={Expected June 2014 – Sep 2014},
    location={Mountain View, CA},
    logo={images/logo/coursera.png},
    description={
Placeholder text designed to have exactly three lines. Three lines describing what you did in this job is just about right for this template. Keep it simple and understandable. Let the details for the interview.
    }
}

\sectionspace % Some whitespace between each position

\createposition{
    company={MUSIXMATCH},
    title={Head Undergrad Research},
    date={Mar 2014 – Present},
    location={Ithaca, NY},
    logo={images/logo/musixmatch.png},
    description={
Placeholder text designed to have exactly three lines. Three lines describing what you did in this job is just about right for this template. Keep it simple and understandable. Let the details for the interview.
    }
}

\sectionspace % Some whitespace between each position

\createposition{
    company={NASA},
    title={Collaborator},
    date={Jan 2014 – Mar 2014},
    location={Ithaca, NY},
    logo={images/logo/nasa.png},
    description={
Placeholder text designed to have exactly three lines. Three lines describing what you did in this job is just about right for this template. Keep it simple and understandable. Let the details for the interview.
    }
}

\sectionspace % Some whitespace between each position

%------------------------------------------------
% Education
%------------------------------------------------

\rightsection{Education}{\educationicon}

\createposition{
    company={LA SAPIENZA},
    title={MSc. Artificial Intelligence and Robotics},
    date={June 2014 – Sep 2014},
    location={Rome, Italy},
    logo={images/logo/sapienza.jpg},
    description={
Placeholder text designed to have exactly three lines. Three lines describing what you did in this job is just about right for this template. Keep it simple and understandable. Let the details for the interview.
    }
}

\sectionspace % Some whitespace between each position

\createposition{
    company={ALMA MATER STUDIORUM},
    title={BSc. Automation Engineering},
    date={June 2014 – Sep 2014},
    location={Bologna, Italy},
    logo={images/logo/almamater.png},
    description={
Placeholder text designed to have exactly three lines. Three lines describing what you did in this job is just about right for this template. Keep it simple and understandable. Let the details for the interview.
    }
}

\sectionspace % Some whitespace after the section

%------------------------------------------------
% Awards and Certificates
%------------------------------------------------

\rightsection{Awards and Certificates}{\awardscertificatesicon}

\dateitem{2014}{KPCB Engineering Fellow}
\dateitem{2014}{Google Code Jam, Qualification Round}
\dateitem{2014}{Microsoft Coding Competition, Cornell}
\dateitem{2013}{Jump Trading Challenge Finalist}

\vspace{-12pt} % For some reason there is a big space between this section and the following, so reduce it
\sectionspace % Some whitespace after the section

%------------------------------------------------
% Publications
%------------------------------------------------

\rightsection{Publications}{\publicationsicon}

% match the same width of positions minipage
\begin{minipage}{\positionwidth} 

\bibliography{sample}

\end{minipage}

\sectionspace % Some whitespace after the section

%----------------------------------------------------------------------------------------

\end{minipage} % The end of the right column

%----------------------------------------------------------------------------------------

\end{document}